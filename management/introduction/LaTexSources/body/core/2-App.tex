\section{Application}


    Tout ce qui a été raconté jusqu'ici je l'ai découvert et expérimenté au
cours du développement de QuanTrade. On va donc passer à la pratique en
détaillant le fonctionnement du programme dans son état actuel et établir
ce qui constitue les bases (du lourd les bases) du projet.

    Deux précisions avant d'avancer: tout ce qui va être évoqué est
fonctionnel, et est autant valable en temps réel qu'en backtest sur les
années précédentes. \newline
Le principe du programme est de surveiller régulièrement l'état du
marché et procéder à des calculs pour détecter les opportunités d'achat
ou de vente, de passer ces signaux à un manager de portefeuille qui
proposera les ordres à émettre, selon diverses optimisations. Ce
fonctionnement de base est hautement configurable: les algorithmes, le
marché, la monnaie, le type de données utilisées\ldots tout.\newline

A partir de là plusieurs interactions utilisateurs sont possibles. \newline

\begin{itemize}

  \item La première est de simplement choisir parmi les listes les
    paramètres que l'on souhaite et regarder le système faire le sale
    boulot à sa place depuis un client web temps réel spécialisé en
    visualisation de donnée (Pour ceux qui ont suivi on commence à faire
    allusion à l'étape 5 de la partie précédente).\newline

  \item Plus puissant et plus geek, il est possible d'écrire sois même un
    certains nombres de modules qui s'intégreront automatiquement au
    système. Ainsi le programme met à dispositions de puissants outils pour
    personnaliser les stratégies algorithmiques, les gestions de
    portefeuille, les optimisateurs de paramètres, les sources de données
    utilisées et bien sûr la façon de les combiner. L'ensemble peut donc
    également être vu comme un véritable labo de trading quantitatif
    (toujours pour ceux qui suivent l'histoire des sources de données
    est le point d'entrée des étapes 1, 2 et 3).\newline

  \item Enfin la place de contributeur amène à intervenir sur l'ensemble du
    code et de la stratégie de développement du logiciel.\newline

\end{itemize}

    Je vais répondre tout de suite à des questions qu'on m'a déjà posées.
L'objectif n'est pas de passer automatiquement les ordres. C'est facilement
envisageable avec les intermédiaires étrangers (les français n'aiment pas trop
ça) mais ce serait dommage de se priver du cerveau humain comme formidable
outil de décision et d'appréhension. De plus ne se positionnant pas sur de
l'hyper-trading, on pourra sans problème réfléchir trèèès lentement. Par
conséquent l'argent géré par le logiciel est virtuel, du virtuel indexé sur
le portefeuille réel qu'il vous appartient d'ouvrir chez un broker et de
gérer selon les saintes suggestions du logiciel.\newline
\newline


\newpage
Plus précisément et plus techniquement, le programme est capable :\newline

\begin{itemize}

  \item d'interfacer R avec le code python du squelette. R est
    le langage le plus puissant en matière de traitement statistique. En
    outre il est extrêmement utilisé, notamment en finance, et il devient
    donc possible de simplement plugger ce que les meilleurs scientifiques
    R\&D proposent en matière d'analyse et d'optimisation boursière (au
    passage on rejoint encore une des étapes de la partie 1, la 6:
    communauté).\newline

  \item de communiquer de manière transparente entre ses modules et
    avec l'utilisateur, en local comme en réseau, de manière totalement
    transparente. Cela repose sur un système de message très simple
    d'utilisation qui permet d'étendre le système par des sous-systèmes
    indépendants (tant qu'ils respectent le protocole de
    communication)\newline

  \item d'archiver les études et les données dans une base MySQL
    pour permettre de capitaliser les résultats et ainsi mener des recherches
    sur de larges échelles. \newline

  \item de proposer une analyse complète des performances des trades, avec
    l'aide d'un puissant générateur graphique et d'une interface web dédiée
    aux expériences.\newline

  \item d'envoyer des notifications sur les smartphones lorsque des
    suggestions sont disponibles ou que de lourds calculs sont
    terminés.\newline

  \item d'optimiser génétiquement les algorithmes écrits par
    l'utilisateur.\newline

  \item d'être commandé vocalement et de répondre (bon\ldots petit goodie datant
    des heures ``creuses'' d'Alten\ldots)\newline
    \newline

\end{itemize}

En outre tout une librairie quantitative est disponible, de même qu'un
certains nombres d'algorithmes prédictifs, d'analyse de tendance sur
données déstructurées.\newline
Et enfin de nombreuses ressources amassées par votre serviteur qui
connaissait quedal il y a 4 mois: livres, notes, liens,
tout ce qu'il faut pour devenir un ninja de tous les domaines cités, les
projets open-source à intégrer d'urgence dans le programme, une liste
indécente d'idées et de concepts à explorer pour littéralement hacker les
marchés financiers!\newline


% section Application (end)
